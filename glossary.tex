\newglossaryentry{totalfunction}
{
  name={total function},
  description={todo},
}
\newglossaryentry{purefunction}
{
  name={pure function},
  description={A function which only depends on its inputs; it is side-effect free/referentially transparent},
}
\newglossaryentry{monad}
{
  name={monad},
  description={ A monad is a structure which represents computations and allows chaining various operations together. 
                In the Haskell language it is used often to abstract stateful computations, but has many different applications.}
}
\newglossaryentry{arrow}
{
  name={arrow},
  description={ A arrow is a structure like \glspl{monad}, but more abstract.  
                todo: better stuff} 
}
\newglossaryentry{clash}
{
  name={C$\lambda$aSH},
  description={ \gls{caes} Language for Synchronous Hardware}
} 
\newglossaryentry{turingcomplete}
{
  name={Turing complete},
  description={ A language is called Turing complete when it can be used to simulate a single taped \gls{turingmachine}.}
}
\newglossaryentry{turingmachine}
{
  name={Turing machine},
  description={ A Turing machine is a hypothetical machine which can be used as a model of computation.}
}
\newacronym{vhsic}{VHSIC}{Very High Speed Intergrated Circuit}
\newacronym{vhdl}{VHDL}{\gls{vhsic} Hardware Description Language}
\newacronym{fpga}{FPGA}{Field-Programmable Gate Array}
\newacronym{hdl}{HDL}{Hardware Description Language}
\newacronym{ghc}{GHC}{Glasgow Haskell Compiler}
\newacronym{forsyde}{ForSyDe}{Formal System Design}
\newacronym{edsl}{EDSL}{Embedded Domain Specific Language}
\newacronym{dsl}{DSL}{Domain Specific Language}
\newacronym{th}{TH}{Template Haskell}
\newacronym{ast}{AST}{Abstract Syntax Tree}
\newacronym{adt}{ADT}{Algebraic Data Type}
\newacronym{smt}{SMT}{Satisfiability Modulo Theories}
\newacronym{caes}{CAES}{Computer Architecture for Embedded Systems}
\newacronym{psl}{PSL}{Property Specification Language}
\newacronym{dsp}{DSP}{Digital Signal Proccessing}
\newacronym{qq}{Quasi-Quotation}{Quasi-Quotations}
\makeglossary
